% !TEX root = ../document.tex

\documentclass{subfiles}


\newenvironment{abstractpage}
  {\cleardoublepage\vspace*{\fill}\thispagestyle{empty}}
  {\vfill\cleardoublepage}
\newenvironment{abstract-lang}[1]
  {\bigskip\selectlanguage{#1}%
   \begin{center}\bfseries\abstractname\end{center}}
  {\par\bigskip}


\begin{document}

  \begin{abstractpage}
    \addcontentsline{toc}{chapter}{\protect\numberline{}Resumen}
    \begin{abstract-lang}{english}
      In this document, a study about the \emph{Dial-a-Ride} problem is carried out, on which it is analyzed in detail from a formal perspective, using its formulation using the \emph{Mixed Integer Linear Programming (MILP)} paradigm, through which its relationship with other \emph{Vehicle Route Problems (VRP)} is discussed. In addition, a description on the most popular resolution methods to solve the \emph{Dial-a-Ride} problem is carried out, starting with the \emph{Exact Methods} to later focus on those based on \emph{(Meta-)Heuristics}. Finally, an initial version of the \emph{jinete} library is presented as a suite of methods for solving the problem, the implementation of which has been carried out as an additional task of this work.
    \end{abstract-lang}
    \begin{abstract-lang}{spanish}
      En este documento se lleva a cabo un estudio acerca del problema \emph{Dial-a-Ride}, sobre el cual es analizado en detalle desde una perspectiva formal, utilizando como herramienta de apoyo su formulación mediante el paradigma de \emph{Programación Lineal Mixta Entera (MILP)}, a través de la cual se discute su relación con otros \emph{Problemas de Rutas de Vehículos (VRP)}. Además, se lleva a cabo una descripción sobre los métodos de resolución más populares para resolver el problema \emph{Dial-a-Ride}, partiendo por los \emph{Métodos Exactos} para posteriormente enfocarse en aquellos basados en \emph{(Meta-)Heurísticas}. Finalmente, se presenta versión inicial de la biblioteca \emph{jinete} como suite de métodos de resolución del problema, cuya implementación se ha llevado a cabo como tarea adiccional de este trabajo.
    \end{abstract-lang}

    \bigskip
    \centering
    Este trabajo puede ser consultado a través del siguiente enlace: \url{https://github.com/garciparedes/tfg-dial-a-ride-problem}

    \bigskip
    La biblioteca \texttt{jinete} se encuentra publicada en el siguiente enlace: \url{https://github.com/garciparedes/jinete/}

  \end{abstractpage}

\end{document}
