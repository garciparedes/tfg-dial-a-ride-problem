% !TEX root = ../../document.tex

\documentclass{subfiles}

\begin{document}

  \chapter{Conclusiones Generales y Próximos Pasos}
  \label{chap:conclusions}

    \section{Conclusiones Generales}
    \label{sec:conclusion_general_conclusions}

      \paragraph{}
      Es bien conocido que la realización de proyectos cuya completitud conlleva una duración relativamente larga (al menos 1 mes de dedicación) presentan un conjunto de retos organizativos que en muchas ocasiones tienden a quedar relegados a un plano secundario en lo relativo a la atención que se les presta frente a otros temas más cercanos al cometido final del proyecto. No obstante, es necesario remarcar que dichos retos organizativos son, en muchas ocasiones, uno de los factores más relevantes para que los objetivos planteados inicialmente sean alcanzados de manera satisfactoria o, al menos con un mínimo grado de completitud). En cuanto a los aprendizajes adquiridos a lo largo del trabajo realizado, este es uno de los puntos a destacar. En muchas ocasiones, este tipo de temas relacionados con la organización de proyectos se presentan como herramientas muy importantes desde la perspectiva empresarial, donde se plantean como la solución para mantener un alto grado de visibilidad entre el conjunto de personas que llevan a cabo un proyecto y el cliente o inversor que desea conocer la situación del mismo. Sin embargo, en muchas ocasiones, dicha importancia en lo relacionado con la planificación de tareas organización del tiempo dedicado y otros factores relacionados no se transmite de la misma manera en trabajos de carácter más académico o incluso orientados a la investigación. Como es natural, dichos trabajos no son sencillos en lo relativo a estas cuestiones por la alta incertidumbre que muchas veces en los temas que plantean (tan solo es necesario pensar en la diferencia entre estimar el tiempo necesario hasta conseguir una estrategia de resolución que supere el \emph{estado del arte} actual frente a la inclusión de una nueva funcionalidad en una aplicación móvil, cuyo alcance es bien conocido y del que en cierta manera se tiene un conocimiento previo por la duración de tareas pasadas). Por lo tanto, una de las primeras conclusiones alcanzadas durante la realización del trabajo es la importancia en la definición inicial de un plan de estudio bien definido.

      \paragraph{}
      El siguiente punto a remarcar está relacionado con el aprendizaje sobre la dificultad intrínseca de llevar a cabo un proyecto relacionado con el estudio, análisis y posterior planteamiento e implementación de \emph{estrategias de resolución} para \emph{problemas de optimización combinatoria}, lo cual se considera una experiencia muy enriquecedora, tanto desde el punto de vista de los conocimientos adquiridos durante el proceso desde el punto de vista de los pasos con los que proceder y el grado de profundidad adquirido en algunos de los temas tratados, como desde el punto de vista de la experiencia que en futuras ocasiones podrá ser aplicada en proyectos de naturaleza similar. Dentro de este marco, cabe destacar la capacidad para valorar la calidad de los resultados alcanzados conseguida a lo largo del desarrollo del trabajo. Un ejemplo de esto es la constante búsqueda de los \say{puntos débiles} del trabajo para enfocar los esfuerzos restantes sobre dichas partes con la esperanza de mejorar la calidad global del mismo.

      \paragraph{}
      En los anteriores párrafos se han llevado a cabo una serie de comentarios que describen las conclusines alcanzadas del trabajo, pero desde una perspectiva del desarrollo del mismo. Sin embargo, también se han obtenido otra serie de conclusiones más cercanas al tema que se ha tratado durante el mismo. El resto del apartado se enfoca en este tipo de conclusiones para después finalizar recordando los objetivos planteados en el capítulo introductorio así como la manera en que estos han sido alcanzados.

      \paragraph{}
      La conclusión principal alcanzada con respecto al estudio del problema \emph{Dial-a-Ride} ha sido su no trivialidad. Esta se puede apreciar en el estudio del modelo de \emph{programación lineal mixta y entera (MILP)} y las restricciones que lo componen, aunque también se puede apreciar al estudiar las estrategias de resolución. Sin embargo, el punto determinante en este sentido a sido el desarrollo de la implementación capaz de generar soluciones factibles. En este sentido, características propias de los problemas de rutas (procedentes del \emph{problema del viajane}) como la secuenciación y unicidad de vértices en los caminos generados sobre el grafo de localizaciones ya representa una cierta dificultad, la cual se extiende al tratar con varios vehículos de manera simultanea. Desde un punto de vista más cercano a la dificultad durante el proceso de optimización, surgen las restricciones relativas a las ventanas temporales que restringen los intervalos de tiempo para completar ciertas tareas, por lo que, en este caso la componente temporal cobra relevancia en el problema. En esta línea, aunque en otra dimensión del problema, se encuentra la necesidad de mantener un control detallado sobre la capacidad de cada vehículo en cada punto del proceso de optimización. Siguiendo en la dirección de un incremento ascendente en lo relativo a la complejidad del problema, el siguiente factor determinante es el relacionado con las tareas pareadas. Es decir, las tareas que conforman un viaje mediante las correspondientes paradas de recogida y de entrega. Este característica implica el reto de tener que ser capaz de mantener de alguna forma la relación de dependencia entre dichas paradas en una misma ruta a la vez que se permite la inclusión de otras paradas intermedias relativas a otros viajes completados en la ruta. En cuanto a la última característica de gran relevancia que se comenta, relacionada con la limitación del tiempo máximo de viaje (es decir, desde que se sale de la parada de recogida hasta que se alcanza la parada de entrega) parece fácil darse cuenta de que la inclusión de esta funcionalidad en la implementación realizada no es muy compleja tomando como referencia la necesidad de cumplir otras como la de imponer que una parada de recogida siempre sea anterior a una de entrega. Sin embargo, el impacto que la limitación de la duración máxima del viaje tiene sobre el incremento de la dificultad desde el punto de vista de del proceso de optimización es muy elevado. Esto es porque complica en mayor medida el espacio de búsqueda de soluciones, el cuál restringe a estructura muy características, lo que requiere técnicas con un alto grado de sofisticación para su correcta resolución.

      \paragraph{}
      Desde el punto de vista de los resultados obtenidos tras la realización del experimento de validación de los mismos, estos han servido para comprender el grado de dificultad real que presenta la tarea de generar métodos de resolución desde el planteamiento inicial de estos hasta el correspondiente proceso de desarrollo con las dificultades inherentes al problema \emph{Dial-a-Ride}. No obstante, en este punto es importante remarcar que no se sabe si las capacidades de cómputo destinadas a la resolución del problema han sido comparables a las utilizadas en otros trabajos que obtienen resultados mucho más positivos.

      \paragraph{}
      En lo relativo a los contenidos marcados como objetivos en el \cref{sec:introduction_objectives}, a continuación se procede con un breve comentario acerca de cómo estos han sido abordados a lo largo del trabajo: El objetivo relacionado con \emph{Analizar el problema Dial-a-Ride desde una perspectiva matemática} se ha alcanzado con un nivel de profundidad satisfactorio de tal manera que a lo largo del desarrollo del trabajo se haya podido comprender conceptos relacionados con dicho modelo de una forma adecuada. En cuanto a \emph{Implementar la formulación del mismo en algún sistema de resolución basada en Programación Lineal Mixta y Entera (MILP)}, este objetivo se ha llevado a cabo de una manera un tanto diferente de la esperada dado que, en lugar de dedicar esfuerzos en probar la formulación del modelo directamente sobre un solver como por ejemplo \emph{XPRESS} o \emph{CPLEX}, se ha preferido focalizar los esfuerzos en la utilización de \emph{PuLP}, que a partir de una sintaxis común, es capaz de llevar a cabo la ejecución de instancias en una gran variedad de solvers (incluyento los más populares). El objetivo de \emph{Estudiar los métodos de resolución basados en (meta-)heurísticas más relevantes para resolver el problema} se ha llevado a cabo mediante el análisis bibliográfico relacionado y a su vez el proceso de reflexión necesario para después poder documentarlo en el capítulo destinado a dicho tema. Dicho objetivo además se ha reforzado a lo largo del cumplimiento de \emph{Llevar a cabo una implementación propia de alguna de las estrategias de resolución disponibles}, el cual se ha llevado a cabo a través del diseño e implementación de la biblioteca \emph{jinete}, durante el cual se han comprendido los puntos más críticos del proceso de resolución y además se han tratado de aplicar ciertas soluciones para contrarrestar dichas dificultades. Finalmente, el objetivo de \emph{Diseñar una estrategia de evaluación de resultados de manera metódica que permita hacer una valoración objetiva de los mismos} se ha llevado a cabo utilizando como referencia tareas aquellos llevados a cabo por otros autores en situaciones similares de validación de resultados en este tipo de problemas. En cuanto a los objetivos de carácter más troncal, estos se han ido llevando a cabo de manera inherente durante todo el proceso de realización del trabajo.

    \section{Próximos Pasos}
    \label{sec:conclusion_next_steps}

      \paragraph{}
      Como punto final del trabajo, se enumera una serie de posibles tareas que guardan una relación cercana con las tareas realizadas durante el proceso de completitud de este. Estas representan tanto inquietudes surgidas durante el análisis del problema y el desarrollo de la implementación, como pasos naturales en un proceso de mayor profundización en el tema.

      \begin{itemize}

        \item \emph{Llevar a cabo un ánalisis más exhaustivo del modelo mátemático del problema}

        \item \emph{Explorar las características concretas de la versión dinámica del problema Dial-a-Ride, cómo este se estudia desde una perspectiva matemática y cuáles son los enfoques de resolución más destacados}

        \item \emph{Estudiar posibles variaciones del propio problema, que permitan llevar a cabo una modelización más verosimil de las necesidades actuales en lo relativo a este, tales como la gestión de combustible (con las características propias de carga de vehículos electricos) u otras variantes de interés}

        \item \emph{Analizar las estrategias basadas en modelos híbridos, que combinan técnicas de resolución exactas con componentes de naturaleza heurística}

        \item \emph{Profundizar en mayor medida en los métodos de resolución (meta-)heurísticos más novedosos y con resultados de mejor calidad}

        \item \emph{Continuar con el proceso de mejora de la implementacion de la biblioteca \emph{jinete}, llevando a cabo un análisis en profundidad de los puntos más críticos y mejorando el grado de eficiencia de la misma}

        \item \emph{Implementar métodos más avanzados como \emph{Búsqueda Tabú} de resolución sobre la estructura ya creada en \emph{jinete}}

        \item \emph{Extender el alcance de la implementación a otros problemas de optimización de rutas diferentes a \emph{Dial-a-Ride}}.

        \item \emph{Llevar a cabo una migración completa de la implementación actual a otros lenguajes de programación más enfocados en la computación de alto rendimiento como \emph{Rust} o \emph{C++}}.

      \end{itemize}

\end{document}
