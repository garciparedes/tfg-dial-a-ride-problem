% !TEX root = ../../document.tex

\documentclass{subfiles}

\begin{document}

  \chapter{Introducción}
  \label{chap:introduction}

    \section{Problema Dial-a-Ride}
    \label{sec:introduction_dial_a_ride}

      \paragraph{}
      El problema \emph{Dial-a-Ride} se enmarca dentro del marco de los problemas de generación de rutas de vehículos. Estos consisten, tal y como se puede deducir fácilmente por su nombre, en la tarea de generar un conjunto de rutas o caminos dados unos determinados vehículos, de tal manera que las rutas generadas sean capaces de satisfacer un conjunto de tareas dado. Dichas tareas se caracterizan por poseer como al menos un lugar de realización. Sin embargo, estas también se pueden caracterizar por una cierta cantidad de mercancia que posiblemente haya que desplazar o alguna limitación de naturaleza temporal como horarios de apertura y cierre. Entonces, de la necesidad de satisfacer dichas tareas es de donde surge la necesidad de definir rutas que sean capaces de satisfacerlas. Como es natural, la acción de llevar a cabo un desplazamiento desde un punto a otro del espacio requerirá un determinado coste de ejecución, el cual no será constante, sino que típicamente seguirá una relación de proporcionalidad respecto de alguna medida de distancia. Además, los vehículos disponibles para llevar a cabo las rutas posiblemente presenten ciertas limitaciones, generalmente relacionadas con la capacidad máxima de la que estos disponen (cuando la tarea consiste en mover una cierta mercancia entre determinados puntos), aunque también pueden existir limitaciones relacionadas con la distancia máxima recorrida (y por tanto de combustible o jornada laboral), aunque es fácil imaginar otras limitaciones de naturaleza variada que se pueden dar en este tipo de problemas.

      \paragraph{}
      Es fácil imaginar situaciones reales que se asemejen a lo descrito en el párrafo anterior donde, por ejemple, cierta empresa dedicada al sector del transporte de mercancias por carretera (posiblemente mediante camiones) tenga la necesidad de indicar a los conductores de su flota de vehículos qué mercancias necesitan cargar en estos, así como en qué lugares deben realizar las correspondientes descargas. Por lo comentado hasta ahora, es posible comprender que la realización de dicha tarea organizativa no es sencilla en casos donde la flota de vehículos es elevada y, además, las características concretas del negocio requieren de tener en cuenta ciertas situaciones complejas.

      \paragraph{}
      Desde un punto de vista empresarial, no solo es importante ser capaz de poder llevar a cabo la generación de rutas que deberán llevarse a cabo por la flota de vehículos de la correspondiente compañía, sino que además existe una necesidad monetaria de reducción de costes innecesarios. Es entonces cuando surge la necesidad no solo de estudiar técnicas que sean capaces de generar planificaciones de rutas que sean capaces de satisfacer las necesidades requeridas, sino que además parece razonable que dicha tarea de planificación se lleve a cabo reduciendo cierta medida de calidad, que en muchos casos se termina traduciendo en un valor monetario.

      \paragraph{}
      De lo comentado en el anterior párrafo surgen dos ideas muy interesantes y a la vez bien diferencidas entre sí. En primer lugar se comenta la necesidad de ser capaces de satisfacer ciertas restricciones. Esta tarea se conoce matemáticamente como \emph{Programación con Restricciones (o Constraint Programming)} y la tarea a resolver se limita únicamente a ser capaz de \say{superar} todas las restricciones impuestas por el problema. Como es natural, existen muchos problemas englobables dentro de esta categoría, aunque uno de los ejemplos más notorios de ello se corresponde con el juego \emph{Sudoku}, donde típicamente no se impone ninguna medida de \say{calidad de la solución} más allá de si es o no válida. Volviendo al tema principal, en el párrafo anterior se comenta una segunda idea muy importante (y la cuál representa la base más relevante de este trabajo), la cual trata sobre la necesidad de generar soluciones que no solo sean capaces de satisfacer las restricciones impuestas por el problema, sino que además busca que esta sea \emph{óptima} respecto de una determinada métrica. Este tipo de problemas son conocidos como \emph{Problemas de Optimización (o Optimization Problems)}, de esta forma surgen una gran cantidad de situaciones que se dan en la vida real: desde el problema de organizar la división de una serie de tareas dependientes entre si, dada una cantidad de recursos disponibles tratando de minimizar la duración completa del proceso, hasta la tarea de elegir qué plan de aprovisionamiento permite llevar a cabo el menor incremento de costes para la fabricación de un determinado producto.

      \paragraph{}
      Ligadas a la idea \emph{Problemas de Optimización (o Optimization Problems)} surgen una serie de estructuras matemáticas que permiten modelar la situación a resolver sobre un entorno cerrado, permitiendo una abstracción del contexto real, lo cual muy ventajoso desde una perspectiva técnica. Dichas estructuras matemáticas son las siguientes:
      \begin{itemize}

        \item \textbf{Variables de Decisión}: Se encargan de representar una determianda acción desde el punto de vista de la toma de una decisión, la cual puede basarse desde la cantidad producto que utilizar, hasta la de si llevar a cabo o no una determinada tarea.

        \item \textbf{Restricciones}: Se trata de un conjunto de limitaciones que se dan (posiblemente de manera combinada) entre las variables de decisión utilizadas para representar el problema. Por ejemplo, estas pueden basarse en la limitación máxima de utilización de cierto producto.

        \item \textbf{Función Objetivo}: Es aquella que, dada una configuración concreta de variables de decisión sobre un problema bien definido, es capaz de generar una métrica (posiblemente multidimensional) con la cual valorar positiva (o negativamente) unas soluciones frente a otras. Dependiendo del problema concreto, el objetivo puede ser la minimización o maximización de la métrica obtenida.

      \end{itemize}

      \paragraph{}
      Una vez llevada a cabo la conexión entre los problemas de rutas y las situaciones que se dan en el mundo real, así como realizada una breve introducción acerca de los \emph{Problemas de Optimización}, ya se está en condiciones suficientes como para proceder a una introducción más profunda en el problema \emph{Dial-a-Ride}.

      \paragraph{}
      El problema \textbf{Dial-a-Ride} se corresponde con una variante del \emph{problema de recogidas y entregas con ventanas temporales}. Por lo tanto, antes de entender de qué manera se caracteriza el problema \emph{Dial-a-Ride} es necesario entender en qué se basa este último. Al principio del capítulo se comentaba que los \emph{problemas de rutas de vehículos} se caracterizan principalmente por ser aquellos cuya finalidad es la de generar configuraciones basadas en conjuntos de rutas referidas a los vehículos diponibles, de tal manera que se satisfagan un conjunto de tareas a resolver. Sobre dicho contexto, es fácil ampliar el alcance del problema a \emph{problemas de rutas de vehículos con ventanas temporales}. En este caso, las ventanas temporales se refieren a las limitaciones de tiempo para satisfacer las tareas (y en ciertas ocasiones incluso la utilización de determinados vehículos). Dichas restricciones proporcionan una gran capacidad de representación de situaciones reales, sobre todo en el entorno comercial de servicio al público, donde quizás la disponibilidad de algunos establecimientos no sea continua. Extender el \emph{problema de rutas de vehículos con ventanas temporales} a la idea de \emph{recogidas y entregas}, donde la tarea no consiste en trasladar una cierta mercancia desde o hasta el almacén, sino de moverla de un destino a otro genera un mayor grado de dificultad conceptual. Sin embargo, las posibilidades que esta modelización genera añaden una potencia representativa muy superior. Desde un punto de vista de aplicabilidad sobre situaciones reales, en este caso se permite la liberta de que no haya un único punto con toda la oferta (o demanda), lo que es una característica muy común en el mundo empresarial. Siguiendo con el ejemplo de una empresa de transporte de mercancias, en muchas ocasiones las tareas cotidianas no consisten en proceder al transporte de las correspondientes mercancias a un único almacen, sino que se trata del movimiento de las mismas entre puntos de generación (como productores de materia prima) hasta puntos de consumo (como fábricas que transforman dicha materia prima en un producto elaborado). Dicho ejemplo puede extenderse de manera natural entre las areas de fabricación de productos y los puntos de venta al consumidor, donde existen ocasiones en que la dirección de estos trayectos es bidireccional (movimiento de stock, devoluciones, etc.). El problema \emph{Dial-a-Ride} surge sobre el contexto de los \emph{problemas de recogidas y entregas con ventanas temporales} tal y como se ha dicho anteriormente. Sin embargo, este posee una característica adiccional que permite su uso sobre el contexto del transporte de pasajeros, el cual se conoce como \emph{calidad del servicio}. En este caso, la calidad del servicio se corresponde con la necesidad natural de mantener al pasajero el menor tiempo posible en el vehículo, lo cual cobra especial sentido cuando se trata de trasladar personas que no quieren perder más tiempo del necesario en el trayecto. Este concepto es de vital importancia, dado que en el transporte de mercancias no existe la necesidad de limitar la duración del trayecto dado que el único recurso que estas consumen es de capacidad. Sin embargo, en el contexto de las personas no sería de recibo solicitar un viaje que normalmente conllevaría $30$ minutos y, a costa de generar una reducción de costes, este se alarge hasta las $3$ horas ya que, como es natural, la persona obtendría un grado de insatisfacción muy elevado.

      \paragraph{}
      Una vez comprendido en qué se basa el problema \emph{Dial-a-Ride}, es fácil proporcionar una serie de ejemplos prácticos que ayuden a contextualizar aún más el problema. En los últimos años han proliferado muchas empresas dedicadas al sector del transporte urbano punto a punto. Algunos ejemplos de ello son \emph{Cabify}, \emph{Uber} o \emph{Lyft}. El modelo de negocio de estas consiste en la solicitud de viajes a partir de una aplicación móvil a partir de la cual se indica el punto de origen (típicamente la posición actual) así como el destino al cual se desea llegar. Entonces, estas se encargan de generar una oferta a satisfacer por partir de su flota de vehículos disponibles a cambio de un determinado precio. Tal y como se puede comprender, este es un caso de aplicación directa del problema \emph{Dial-a-Ride} donde, a partir de la planificación de ruta, se genera un coste (que tenderá a ser minimizado para que el cliente no utilice un servicio de la competencia), pero que a la vez a de tener en cuenta la solicitud de todos los posibles clientes, lo cual puede generar situaciones complejas que requieren de técnicas sofisticadas para dar con la planificación adecuada. El segundo ejemplo consiste en los servicios de comida a domicilio, que operan empresas como \emph{Just Eat}, \emph{Uber Eats}, \emph{Deliveroo} o \emph{Glovo} donde en este caso la tarea consiste en llevar un cierto producto desde un determinado restaurante hasta el hogar de la persona solicitante. En este caso también se da la restricción de \emph{calidad del servicio} transmitida como la temperatura a la cual llega la comida al cliente.

      \paragraph{}
      A modo de reflexión sobre los casos prácticos comentados en el anterior párrafo, el lector se ha podido dar cuenta de que en muy pocas ocasiones estos utilizan algún tipo de estrategia que permita la combinación de trayectos entre viajes para reducir costes (a excepción del servicio \emph{Uber Pool}). A pesar de ello, dicha componente es muy relevante en procesos de resolución de este tipo de problemas, por lo tanto, se cree necesario el estudio y desarrollo de técnicas eficientes que permitan aplicar dichas características a problemas de gran tamaño, sin que la calidad del servicio sea perjudicada.

    \section{Objetivos}
    \label{sec:introduction_objectives}

      \paragraph{}
      [TODO]

    \section{Metodología}
    \label{sec:introduction_metodology}

      \paragraph{}
      [TODO]

\end{document}
