% !TEX root = ../../document.tex

\documentclass{subfiles}

\begin{document}

  \chapter{Formulación del Problema}
  \label{chap:formulation}

    \section{Introducción}
    \label{sec:formulation_introduction}

      \paragraph{}
      El problema de recogidas y entregas (\emph{Pickup and Delivery}), o \emph{PDP} en modo abreviado representa una de las modelizaciones más interesantes en el ámbito de los problemas de \emph{optimización combinatoria}. Esto se debe a la gran cantidad de situaciones del mundo real que pueden ser representadas siguiendo dicho esquema. Sin embargo, antes de profundizar en los aspectos más detallados que caracterizan el problema de recogidas y entregas, es necesario describir el contexto del mismo, así como la clase a la cual pertenece. Una vez se haya completado dicha tarea, se estará en condiciones necesarias para poder describir tanto la versión básica como las extensiones más interesantes, tanto desde el punto de vista de los aspectos matemáticos, como desde la cantidad de situaciones reales que permiten resolver.

      \paragraph{}
      En cuanto a la organización del capítulo, en este apartado se describe de manera detallada el contexto del problema desde el punto de vista de la clase de problemas matemáticos a la que pertenecen (\cref{sec:formulation_optimization_problems,sec:formulation_linear_problems,sec:formulation_combinatorial_problems,sec:formulation_routing_problems}), las características generales que presenta (\cref{sec:formulation_picup_and_delivery_problems}) y algunas de las situaciones de gran relevancia para nuestra sociedad que están siendo resueltas siguiendo la modelización \emph{PDP} (\cref{sec:formulation_applications}).

      [TODO: continuar descripción del resto de apartados del capítulo]

      \paragraph{}
      A continuación se procede a describir la clase de problemas matemáticos a la cual pertenece el \emph{problema de recogidas y entregas}. Dicha descripción se llevará a cabo de fuera hacia dentro, esto es desde la categoría de problemas más amplia hasta la más concreta, pasando por una breve contextualización así como ejemplificación de problemas similares.

      \subsection{Problemas de Optimización}
      \label{sec:formulation_optimization_problems}

        \paragraph{}
        La clase de \emph{problemas de optimización} representa una de las áreas de investigación más interesante en la actualidad, ya que muchas de las innovaciones obtenidas en dicho campo permiten resolver problemas aplicables al mundo real de manera práctica que antes únicamente podían ser resueltos teóricamente. En concreto, los problemas de optimización son aquellos que se basan en la minimización (o maximización) de una determinada función objetivo (posiblemente vectorial) de manera que se satisfaga un conjunto de restricciones previamente fijadas sobre un conjunto de variables de decisión que afectan mutuamente a la satisfacibilidad de las restricciones y el valor de la función objetivo.

        \paragraph{}
        Dichas variables de decisión pueden ser tanto categóricas como numéricas (discretas o continuas), lo cual genera una gran cantidad de subproblemas diferences (Nótese que las variables categóricas con $k$ niveles diferentes pueden ser representadas de manera sencilla a partir de $k-1$ variables binarias). De la misma manera, tanto el valor de función objetivo como las restricciones pueden tener una naturaleza muy diferente: estas pueden estar formadas por funciones lineales de las variables de decisión, como por complicadas funciones no lineales que complican el proceso de obtención del valor óptimo del problema.

        \paragraph{}
        Muchos de los problemas que resolvemos a diario en nuestra vida cotidiana son en cierta medida \emph{problemas de optimización}, desde qué elementos decidimos añadir a nuestra mochila cada día (basados en restricciones de capacidad, funciones objetivo de utilidad y variables de decisión binarias) hasta el la detección del rostro por nuestros teléfonos móviles para aplicar un filtro de la manera más realista posible en una videollamada (basados en restricciones de forma, funciones objetivo multidimensionales y millones de variables de decisión numéricas).

      \subsection{Problemas de Optimización Lineal}
      \label{sec:formulation_linear_problems}

        \paragraph{}
        [TODO]

      \subsection{Problemas de Optimización Combinatoria}
      \label{sec:formulation_combinatorial_problems}

        \paragraph{}
        [TODO]


      \subsection{Problemas de Rutas}
      \label{sec:formulation_routing_problems}

        \paragraph{}
        [TODO]


      \subsection{Problemas de Recogidas y Entregas}
      \label{sec:formulation_picup_and_delivery_problems}

        \paragraph{}
        [TODO]


      \subsection{Applicaciones Reales}
      \label{sec:formulation_applications}

        \paragraph{}
        [TODO]

    \section{Notación}
    \label{sec:formulation_notation}

      \paragraph{}
      [TODO]

      \begin{itemize}
        \item $V_{i}$: [TODO]
        \item $A_{l}$: [TODO]
        \item $K_{k}$: [TODO]
      \end{itemize}

      \paragraph{}
      [TODO]


    \section{Formulación básica}
    \label{sec:formulation_basic_formulation}

      \paragraph{}
      Modelización basada en \cite{parragh2008survey}. [TODO]


      \begin{eqfloat}
				\begin{equation}
					\begin{array}{rr@{}ll}
  					\text{Minimizar} & \displaystyle\sum\limits_{k \in K}\sum\limits_{(i, j) \in A}  c_{ij}^{k}x_{ij}^{k} &                 & \\
						\text{sujeto a}	 & \displaystyle\sum\limits_{k \in K}\sum\limits_{j: (i, j) \in A} x_{0j}^{k} \ &= 1,                      & \forall i \in P \cup D \\
                             & \displaystyle\sum\limits_{j: (0, j) \in A} x_{i,n+\tilde{n} + 1}^{k}  \ &= 1, 	                          & \forall k \in K \\
                             & \displaystyle\sum\limits_{i: (i, n + \tilde{n} + 1) \in A} x_{ij}^{k} \ &= 1, 	                          & \forall k \in K \\
                             & \displaystyle\sum\limits_{i: (i, j) \in A} x_{ij}^{k} - \sum\limits_{j: (j, i) \in A} x_{ij}^{k} \ &= 0, & \forall j \in P \cup D, \forall k \in K \\
                             & x_{ij}^{k} = 1, \implies B_{j}^{k} \ &\geq B_{i}^{k} + d_{i} + t_{ij}^{k} 	                          & \forall (i, j) \in A, \forall k \in K \\
                             & x_{ij}^{k} = 1, \implies Q_{j}^{k} \ &= Q_{i}^{k} + q_{j} 	                          & \forall (i, j) \in A, \forall k \in K \\
                             & max\{0, q_{i}\} \ &\leq Q_{i}^k 	                          & \forall i \in V, \forall k \in K \\
                             & Q_{i}^k \ &\leq min\{C^{k}, C^{k} + q_{i}\} 	                          & \forall i \in V, \forall k \in K \\
														 &                               	x_{j} 	&\in \{0,1\}, 	                 & \forall j \in \{1,...,n\} \\
					\end{array}
				\end{equation}
				\caption{[TODO]}
				\label{eq:basic_formulation}
			\end{eqfloat}

      \paragraph{}
      [TODO]

    \section{Restricciones Addicionales}
    \label{sec:formulation_aditional_restrictions}

      \paragraph{}
      [TODO]

      \subsection{Ventanas Temporales}
      \label{sec:formulation_time_window_restrictions}

        \paragraph{}
        [TODO]

      \subsection{Duración de viaje}
      \label{sec:formulation_trip_duration_restrictions}

        \paragraph{}
        [TODO]

      \subsection{Duración de ruta}
      \label{sec:formulation_route_duration_restrictions}

        \paragraph{}
        [TODO]

    \section{Funciones Objetivo}
    \label{sec:formulation_objective_functions}
      [TODO]

      \subsection{Pickup and Delivery}
      \label{sec:formulation_pickup_and_delivery}

        \paragraph{}
        [TODO]

      \subsection{Dial a Ride}
      \label{sec:formulation_dial_a_ride}

        \paragraph{}
        [TODO]

      \subsection{Taxi Sharing}
      \label{sec:formulation_taxi_sharing}

        \paragraph{}
        [TODO]

    \section{Tiempo Real}
    \label{sec:formulation_real_time}

      \paragraph{}
      [TODO]

    \section{Conclusiones}
    \label{sec:formulation_conclusions}

      \paragraph{}
      [TODO]

\end{document}
