% !TEX root = ../../document.tex

\documentclass{subfiles}

\begin{document}

  \chapter{Formulación del Problema}
  \label{chap:formulation}

    \section{Introducción}
    \label{sec:formulation_introduction}

      \paragraph{}
      El problema de recogidas y entregas (\emph{Pickup and Delivery}), o \emph{PDP} en modo abreviado representa una de las modelizaciones más interesantes en el ámbito de los problemas de \emph{optimización combinatoria}. Esto se debe a la gran cantidad de situaciones del mundo real que pueden ser representadas siguiendo dicho esquema. Sin embargo, antes de profundizar en los aspectos más detallados que caracterizan el problema de recogidas y entregas, es necesario describir el contexto del mismo, así como la clase a la cual pertenece. Una vez se haya completado dicha tarea, se estará en condiciones necesarias para poder describir tanto la versión básica como las extensiones más interesantes, tanto desde el punto de vista de los aspectos matemáticos, como desde la cantidad de situaciones reales que permiten resolver.

      \paragraph{}
      En cuanto a la organización del capítulo, en esta sección se describe de manera detallada el contexto del problema desde el punto de vista de la clase de problemas matemáticos a la que pertenecen (\autoref{sec:formulation_optimization_problems}, \autoref{sec:formulation_combinatorial_problems} y \autoref{sec:formulation_routing_problems}), las características generales que presenta (\autoref{sec:formulation_picup_and_delivery_problems}) y algunas de las situaciones de gran relevancia para nuestra sociedad que están siendo resueltas siguiendo la modelización \emph{PDP} (\autoref{sec:formulation_applications}).

      \subsection{Problemas de Optimización}
      \label{sec:formulation_optimization_problems}

        \paragraph{}
        [TODO]


      \subsection{Problemas de Optimización Combinatoria}
      \label{sec:formulation_combinatorial_problems}

        \paragraph{}
        [TODO]


      \subsection{Problemas de Rutas}
      \label{sec:formulation_routing_problems}

        \paragraph{}
        [TODO]


      \subsection{Problemas de Recogidas y Entregas}
      \label{sec:formulation_picup_and_delivery_problems}

        \paragraph{}
        [TODO]


      \subsection{Applicaciones Reales}
      \label{sec:formulation_applications}

        \paragraph{}
        [TODO]

    \section{Notación}
    \label{sec:formulation_notation}

      \paragraph{}
      [TODO]

      \begin{itemize}
        \item $V_{i}$: [TODO]
        \item $A_{l}$: [TODO]
        \item $K_{k}$: [TODO]
      \end{itemize}

      \paragraph{}
      [TODO]


    \section{Formulación básica}
    \label{sec:formulation_basic_formulation}

      \paragraph{}
      Modelización basada en \cite{parragh2008survey}. [TODO]


      \begin{eqfloat}
				\begin{equation}
					\begin{array}{rr@{}ll}
  					\text{Minimizar} & \displaystyle\sum\limits_{k \in K}\sum\limits_{(i, j) \in A}  c_{ij}^{k}x_{ij}^{k} &                 & \\
						\text{sujeto a}	 & \displaystyle\sum\limits_{k \in K}\sum\limits_{j: (i, j) \in A} x_{0j}^{k} \ &= 1,                      & \forall i \in P \cup D \\
                             & \displaystyle\sum\limits_{j: (0, j) \in A} x_{i,n+\tilde{n} + 1}^{k}  \ &= 1, 	                          & \forall k \in K \\
                             & \displaystyle\sum\limits_{i: (i, n + \tilde{n} + 1) \in A} x_{ij}^{k} \ &= 1, 	                          & \forall k \in K \\
                             & \displaystyle\sum\limits_{i: (i, j) \in A} x_{ij}^{k} - \sum\limits_{j: (j, i) \in A} x_{ij}^{k} \ &= 0, & \forall j \in P \cup D, \forall k \in K \\
                             & x_{ij}^{k} = 1, \implies B_{j}^{k} \ &\geq B_{i}^{k} + d_{i} + t_{ij}^{k} 	                          & \forall (i, j) \in A, \forall k \in K \\
                             & x_{ij}^{k} = 1, \implies Q_{j}^{k} \ &= Q_{i}^{k} + q_{j} 	                          & \forall (i, j) \in A, \forall k \in K \\
                             & max\{0, q_{i}\} \ &\leq Q_{i}^k 	                          & \forall i \in V, \forall k \in K \\
                             & Q_{i}^k \ &\leq min\{C^{k}, C^{k} + q_{i}\} 	                          & \forall i \in V, \forall k \in K \\
														 &                               	x_{j} 	&\in \{0,1\}, 	                 & \forall j \in \{1,...,n\} \\
					\end{array}
				\end{equation}
				\caption{[TODO]}
				\label{eq:basic_formulation}
			\end{eqfloat}

      \paragraph{}
      [TODO]

    \section{Restricciones Addicionales}
    \label{sec:formulation_aditional_restrictions}

      \paragraph{}
      [TODO]

      \subsection{Ventanas Temporales}
      \label{sec:formulation_time_window_restrictions}

        \paragraph{}
        [TODO]

      \subsection{Duración de viaje}
      \label{sec:formulation_trip_duration_restrictions}

        \paragraph{}
        [TODO]

      \subsection{Duración de ruta}
      \label{sec:formulation_route_duration_restrictions}

        \paragraph{}
        [TODO]

    \section{Funciones Objetivo}
    \label{sec:formulation_objective_functions}
      [TODO]

      \subsection{Pickup and Delivery}
      \label{sec:formulation_pickup_and_delivery}

        \paragraph{}
        [TODO]

      \subsection{Dial a Ride}
      \label{sec:formulation_dial_a_ride}

        \paragraph{}
        [TODO]

      \subsection{Taxi Sharing}
      \label{sec:formulation_taxi_sharing}

        \paragraph{}
        [TODO]

    \section{Tiempo Real}
    \label{sec:formulation_real_time}

      \paragraph{}
      [TODO]

    \section{Conclusiones}
    \label{sec:formulation_conclusions}

      \paragraph{}
      [TODO]

\end{document}
