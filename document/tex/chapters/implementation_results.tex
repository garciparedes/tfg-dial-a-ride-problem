% !TEX root = ../../document.tex

\documentclass{subfiles}

\begin{document}

  \chapter{Implementación y Resultados}
  \label{chap:implementation_results}

    \section{Introducción}
    \label{sec:implementation_results_introduction}

      \paragraph{}
      Hasta ahora, el objetivo principal de este documento se ha orientado en la descripción y análisis del problema \emph{Dial-a-Ride} desde una perspectiva principalmente teórica, incluyendo una descripción detallada centrada en la \emph{Formulación del Problema} en el \cref{chap:formulation} así como un análisis sobre los \emph{Métodos de Resolución} disponibles para el problema en el \cref{chap:solving}. A lo largo de dichos capítulos se ha podido apreciar tanto las dificultades inherentes al propio problema por las características propias que este presenta, tales como el tiempo máximo de duración de trayecto (calidad del servicio), así como otros factores relacionados con la complejidad computacional de resolver un problema de esta naturaleza (englobado en la clase de problemas \emph{NP}).

      \paragraph{}
      Sin embargo, el este capítulo representa un cambio de perspectiva del problema, estando centrado más en los detalles de implementación de una biblioteca bautizada como \texttt{jinete} y desarrollada sobre el lenguaje \emph{Python}, cuyo enfoque consiste en proporcionar una serie de herramientas que sobre las cuales ser construir métodos de resolución de problemas de rutas. En una primera implementación, el desarrollo de dicha biblioteca se ha orientado principalmente hacia la resolución del problema \emph{Dial-a-Ride} mediante una estrategía inspirada en la metaheurística \emph{GRASP} descrita en el \cref{sec:solving_grasp}. En el \cref{sec:implementation} se incluye una descripción detallada acerca de la implementación realizada, así como las ideas en que ha inspirado la misma.

      \paragraph{}
      Para demostrar el funcionamiento de la implementación realizada, así como proporcionar una ejemplificación de su uso, se han utilizado una serie de instancias del problema disponibles en la web de manera pública, que han sido utilizados a modo de \emph{benchmark} en una gran cantidad de documentos científicos de gran reconocimiento. El \cref{sec:results} se destina a dicho cometido. 
      
      \paragraph{}
      Por último, en el \cref{sec:implementation_results_conclusions} se expone una conclusión relacionada con el proceso de implementación de estrategias de resolución del problema \emph{Dial-a-Ride}, indicando las principales debilidades de la implementación actual, así como sus posibles puntos de mejora.

    \section{Implementación}
    \label{sec:implementation}
      
      \paragraph{}
      Tal y como se menciona en la introducción del capítulo, este apartado se dedica integramente a comentar la implementación realizada durante el desarrollo de este trabajo, donde se describen de manera detallada desde los objetivos iniciales (en el \cref{sec:implementation_objectives}), continuando con una visión de alto nivel de la implementación (en el \cref{sec:implementation_high_level_vision}) para finalizar describiendo los principales componentes (en el \cref{sec:implementation_components}). Además, a lo largo de la descripción se trata de presentar una breve discusión acerca de las distintas decisiones de diseño tomadas a lo largo del proceso, tratando de destacar las principales ventajas e inconvenientes de cada una de ellas.

      \subsection{Objetivos}
      \label{sec:implementation_objectives}

        \paragraph{}
        El objetivo principal de la implementación realizada puede ser resumido en una frase: \textbf{Razonar en detalle la implementación de un sistema capaz de generar soluciones válidas (y lo más cercanas al valor óptimo posible) para el problema Dial-a-Ride}. Sin embargo, tal y como se puede apreciar, dicha frase es de carácter muy general y puede abarcar una gran cantidad de tareas a realizar, desde el estudio y adaptación de restricciones definidas como desigualdades en un modelo lineal para que estas después sean implementadas sobre uno de los sistemas comerciales más populares como \texttt{FICO Xpress Mosel} o \texttt{IBM CPLEX}, hasta el diseño de alguna metaheurística por completo en algún lenguaje como \texttt{C++}, \texttt{Rust} o \texttt{Python}. Tal y como se puede intuir, cualquiera de estas tareas puede llegar a ser por si sola motivo de una trabajo de tesis con el conveniente grado de profundización en la materia. Sin embargo, dado que en este caso el trabajo se ha llevado a cabo sobre el contexto de un \emph{Trabajo de Fin de Grado} (con sus correspondientes limitaciones temporales), la implementación que se ha llevado a cabo consiste en una tarea intermedia entre estos dos caminos, a costa de un menor grado de profundidad en cada uno de ellos, pero que ha permitido conocer las complicaciones reales que la literatura relacioanda con el problema \emph{Dial-a-Ride} pretende resolver para que las soluciones alcanzadas sean aplicables sobre situaciones reales y no queden en meros ejercicios teóricos.

        \paragraph{}
        Una de las decisiones importantes dede el punto de vista de los objetivos de la implementación es elegir el lenguaje sobre el cuál se va a llevar a cabo. El motivo de sobre la relevancia de dicha decisión desde el punto de vista del objetivo de la misma está intimamente relacionada con el alcance del trabajo. En este caso, los principales factores que han influido en dicha decisión han sido la capacidad de generar una implementación relativamente elaborada en un tiempo relativamente reducido, la experiencia en el propio lenguaje y el ecosistema de herramientas relacionadas compatibles con el mismo. Por contra, se han obviado otros factores como el grado de eficiencia del mismo (el cual es muy importante en este tipo de sistemas) pero que se ha dejado en un segundo nivel por la naturaleza didáctica del mismo. Tras tener en cuenta todos estos factores, la decisión tomada ha sido la de elegir \texttt{Python 3} \cite{rossum1995python} como lenguaje sobre el cual realizar la implementación. Este proporciona la capacidad de llevar a cabo implementaciones relativamente grandes en un periodo de tiempo inferior a la que se requeriría por otros lenguajes más verbosos como \texttt{Java} o \texttt{C++} además de proporcionar una sintaxis con una alta legibilidad, lo cual es una factor muy beneficioso en el contexto de los métodos de resolución de problemas como el \emph{Dial-a-Ride}. Otro de los factores es la experiencia adquirida sobre este lenguaje, lo cual es un factor muy facilitador. Sin embargo, la elección escogida también presenta desventajas, sobre todo de escalabilidad a largo plazo y extensión de la misma desde un contexto didáctico hacia otro más aplicable a situaciones reales. Esto se debe a la ineficiencia que este lenguaje presenta frente a otros con una interfaz a un nivel más bajo, que (a costa de una mayor complejidad de desarrollo) son algunos órdenes de magnitud más rápidos. Este es uno de los factores que se comentarán en los próximos pasos, destacando la posibilidad de llevar a cabo una reimplementación de la biblioteca actual sobre el lenguaje \texttt{Rust}.

        \paragraph{}
        En relación con las ideas expuestas en los anteriores párrafos, y obviando la decisión sobre el lenguaje sobre el cuál se ha llevado a cabo la implementación, uno de los factores más importantes para entender la implementación realizada ha sido la naturaleza de la misma. En este caso, en lugar de tratar de llevar a cabo el desarrollo de una serie de ficheros que permitan resolver únicamente el problema \emph{Dial-a-Ride}, sin tener en cuenta las posibles variantes que este pueda llegar a tener, así como las partes en común que distintos métodos de resolución compartan, en este caso se ha primado el análisis de dichas partes. De este modo se ha conseguido alcanza una estructura en forma de biblioteca, la cual se caracteriza por proporcionar una interfaz de utilización externa, sobre la cuál definir por ejemplo la manera de \say{cargar} el fichero de entrada que contiene la entrada del problema así como la estrategia de resolución que se desea utilizar e incluso el modo en que se exportarán los resultados. Por otra parte, se ha tratado de mantener oculta al exterior toda la complejidad inherente a este tipo de problemas de optimización, lo cuál es un punto a favor para ser utilizado por personas que no sean lo suficentemente expertas en la materia. En relación con la naturaleza en forma de biblioteca, también se ha tratado de priorizar la estructura de composición en los métodos de resolución. Es decir, en lugar de proporcionar una serie de implementaciones bien definidas, se ha pretendido proporcionar una interfaz extensible y que favorezca la composición. Esto se puede apreciar en situaciones como la definición (o eleeción) de los criterios de selección en ciertas estrategias de naturaleza \emph{greedy} o la capacidad de componer de manera personalizada distintas metaheurísticas que requieren de fases de inserción llevadas a cabo por otras posibles heurísticas (o metaheurísticas). Cuando se ejemplifique su utilización esto se podrá apreciar de manera más clara. 

        \paragraph{}
        Una de las partes más importantes a la hora de estudiar \emph{problemas de rutas} como el \emph{Dial-a-Ride} consiste en el análisis del mismo desde un pusto de vista más teórico. Uno de los caminos para llevar a cabo dicha tarea es el estudio de la formulación del mismo como \emph{problema de programación lineal} donde, cómo se ha indicado en el \cref{sec:formulation_real_time}, es necesario definir tanto la naturaleza de una serie de variables de decisión, como las restricciones que estas presentan entre si, como la función objetivo (en este caso de minimización). Para que dicho ejercicio de estudio teórico del problema sea completo, también es una buena práctica la de implementarlo en uno de los solvers de propósito general más populares como \texttt{CPLEX} o \texttt{Mosel} de tal manera que se puedan apreciar los resultados así como comprender las dificultades computacionales desde una perspectiva mucho más cercana. Sin embargo, en este trabajo se ha seguido un enfoque ligeramente distinto para tratar de mantener lo más unificada posible la implementación llevada a cabo en \texttt{Python} con la obtenidas por distintos solvers. Para ello, se ha utilizado una heramienta desarrollada por la \emph{COIN-OR Foundation} cuyo cometido es el de proporcionar una interfaz de comunición entre \texttt{Python} y los solvers más conocidos, la cual se conoce como \texttt{PuLP} \cite{mitchell2011pulp}. Para desenpeñar su tarea, esta biblioteca proporciona una interfaz sobre la cual llevar a cabo la formulación del problema a partir de sentencias como \texttt{[TODO]} (para definir el problema), \texttt{[TODO]} (para definir variables), \texttt{[TODO]} (para definir restricciones) o \texttt{[TODO]} (para proceder a la optimización del problema). Internamente, dicha biblioteca es capaz de comunicarse con una gran cantidad de solvers (en el apartado de resultados se incluye una breve comparativa entre estos). Posteriormente, se ha añadido un nivel de abstracción superior capaz de convertir los resultados de la formulación lineal en conceptos relacionados con los problemas de rutas como \emph{Vehículo}, \emph{Ruta} o \emph{Viaje}. De esta manera es sencillo realizar comparaciones entre métodos de resolución basados en heurísticas frente a otros basados en métodos exactos, además de compartir la misma implementación tanto para la lectura de los datos como para exportación de los resultados.

        \paragraph{}
        Hasta ahora se ha comentado el objetivo de la implementación realizada desde el punto de vista de algunas de las decisiones de implementación tomadas a priori (tales como el lenguaje utilizado o la filosofía seguida durante el proceso), sin embargo, aún no se ha expuesto en detalle la problemática concreta que dicha implementación es capaz de resolver, ni la estrategia seguida para llevarlo a cabo. En cuanto a la problemática concreta que esta biblioteca trata de resolver, es fácil intuir (por ser el tema principal del trabajo) que se trata de ser capaz de generar soluciones factibles para el problema \emph{Dial-a-Ride}. Por contra, la segunda parte de la cuestión requiere de una clarificación en mayor profundad: Por una parte, esta implementación es capaz de generar soluciones basándose en métodos de resolución de problemas lineales gracias a la formulación del problema \emph{Dial-a-Ride} de dicha forma, tal y como se ha comentado en el párrafo anterior. Por otro lado, actualmente la biblioteca es capaz de generar soluciones siguiendo una estrategia metaheurística de naturaleza \emph{GRASP}, lo cual implica, entre otros, la implementación de una algoritmo de inserción \emph{Greedy}, una heurística de \emph{Búsqueda Local} capaz de mejorar una solución previamente generada en este caso, por la estrategia \emph{Greedy}, y la propia lógica de control de la estrategia \emph{GRASP}, encargada de controlar si se cumplen las condiciones suficientes para llevar a cabo un nuevo ciclo de mejora de la solución actual así como otros aspectos relacionados. Es importante remarcar que todos estos componentes pueden ser utilizados añadiendo distintas puntos de aleatoriedad (para así tratar de reducir el alcanzamiento de mínimos locales \say{sin salida}). De forma complementaria a la capacidad de añadir una componente de aleatoriedad, también se incluye la capacidad de apoyarse en un \say{generaror de múltiples soluciones} que es capaz de integrarse en cualquier fase del proceso. Posteriormente se detallará más este componente, el cual se encarga de orquestar la generación de soluciones con distintas componentes de aleatoriedad y posteriormente seleccionar cual de todas ellas debe continuar con el proceso de optimización.

        \paragraph{}
        [TODO: conclude the section]
      
      \subsection{Visión a alto nivel}
      \label{sec:implementation_high_level_vision}

        \paragraph{}
        [TODO]

      \subsection{Componentes}
      \label{sec:implementation_components}

        \paragraph{}
        [TODO]

    \section{Resultados}
    \label{sec:results}

      \paragraph{}
      [TODO]

      \subsection{Definición del experimento}
      \label{sec:results_definition}

        \paragraph{}
        [TODO]

      \subsection{Tablas de resultados}
      \label{sec:results_tables}

        \paragraph{}
        [TODO]

      \subsection{Análisis de resultados}
      \label{sec:results_analysis}

        \paragraph{}
        [TODO]

    \section{Conclusiones}
    \label{sec:implementation_results_conclusions}

      \paragraph{}
      [TODO]

\end{document}
