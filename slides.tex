\documentclass[8pt]{beamer}
\setbeamertemplate{navigation symbols}{}

\usepackage[
  english,
  spanish,
  es-tabla,
  es-nodecimaldot
]{babel}
\selectlanguage{spanish}

\usepackage{dirtytalk}
\usepackage{float}
\usepackage{amsmath}
\usepackage{amssymb}
\usepackage[spanish]{cleveref}
\usepackage{thelinearprogramming}
\usepackage{tabu}

\usepackage[newfloat]{minted}
\setminted[python]{tabsize=1,fontsize=\footnotesize}

\title{Métodos escalables de resolución para el \\ Dial-a-Ride Problem (DARP)}
\subtitle{Sergio García Prado}

\usecolortheme{dove}
\begin{document}

  \begin{frame}
		\titlepage
  \end{frame}

  \begin{frame}{Introducción}
		\say{The Dial-a-Ride Problem (DARP) consists of designing vehicle routes and schedules for n users who specify pickup and delivery requests between origins and destinations. The aim is to plan a set of m minimum cost vehicle routes capable of accommodating as many users as possible, under a set of constraints. The most common example arises in door-to-door transportation for elderly or disabled people.}
    \rightline{{\rm --- Jean-François Cordeau \& Gilbert Laporte}}
  \end{frame}

  \begin{frame}{Formulación}
      \begin{equation*}
        \begin{array}{rr@{}ll}
          \text{Minimizar}
            & \displaystyle\sum\limits_{k \in K}\sum\limits_{j \in V}\sum\limits_{i \in V}  c_{ij}^{k}x_{ij}^{k} &
              & \\
          \text{sujeto a}
            & \displaystyle\sum\limits_{k \in K}\sum\limits_{j \in V} x_{ij}^{k} \ &= 1,
              & \forall i \in P \\
            & \displaystyle\sum\limits_{i \in V} x_{0i}^{k}  \ &= 1,
              & \forall k \in K \\
            & \displaystyle\sum\limits_{i \in V} x_{i,2n+1}^{k}  \ &= 1,
              & \forall k \in K \\
            & \displaystyle\sum\limits_{j \in V} x_{ij}^{k} - \sum\limits_{j \in V} x_{n+i,j}^{k} \ &= 0,
              & \forall i \in P, \forall k \in K \\
            & \displaystyle\sum\limits_{j \in V} x_{ji}^{k} - \sum\limits_{j \in V} x_{ij}^{k} \ &= 0,
              & \forall i \in P \cup D, \forall k \in K \\
            & u_{i}^{k} + d_{i} + t_{ij} - M_{ij}^k (1 - x_{ij}^k) \ &\leq u_{j}^{k},
              & \forall i,j \in V, \forall k \in K \\
            & w_{i}^{k} + q_{j} - W_{ij}^k (1 - x_{ij}^k) \ &\leq w_{j}^{k},
              & \forall i,j \in V, \forall k \in K \\
            & u_{2n+1}^{k} - u_{0}^{k} \ &\leq T_{k},
              & \forall k \in K \\
            & e_{i} \ &\leq u_{i}^{k},
              & \forall i \in V, \forall k \in K \\
            & u_{i}^{k} \ &\leq l_{i},
              & \forall i \in V, \forall k \in K \\
            & t_{i, n + i}^{k} \ &\leq u_{n+i}^{k} - (u_{i}^{k} + d_{i}),
              & \forall i \in P, \forall k \in K \\
            & u_{n+i}^{k} - (u_{i}^{k} + d_{i}) \ &\leq L_{k},
              & \forall i \in P, \forall k \in K \\
            & max\{0, q_{i}\} \ &\leq w_{i}^{k},
              & \forall i \in V, \forall k \in K \\
            & w_{i}^{k} \ &\leq Q_{k} + min\{0, q_{i}\},
              & \forall i \in V, \forall k \in K \\
            & x_{ij}^k \ &\in \{0,1\},
              & \forall i,j \in V, \forall k \in K \\
        \end{array}
      \end{equation*}
  \end{frame}

  \begin{frame}{Métodos de Resolución}
    \begin{itemize}
      \item Exactos:
        \begin{itemize}
          \item \emph{Branch and Bound}: Se basa en la resolución una versión relajada del problema para después añadir restricciones adiccionales (valores binarios, etc.) a partir de un proceso de ramificación.
        \end{itemize}
      \item Heurísticos:
        \begin{itemize}
          \item \emph{Fuerza Bruta}: Mediante un proceso iterativo probar todas las posibles combinaciones de soluciones hasta llegar a la mejor solución.
          \item \emph{Selección Aleatoria}: Generación iterativa de soluciones incrementales tomando decisiones aleatorias en cada paso.
          \item \emph{Greedy}: Generación iterativa de soluciones incrementales tomando la mejor decisión en cada paso del algoritmo con la esperanza de que la solución final sea lo suficientemente buena.
          \item \emph{Clarke and Wright}: Algoritmo clásico de generación de rutas, partiendo de una por viaje para después realizar combinaciones entre estas mientras sea posible.
          \item \emph{Búsqueda Local}: Estrategia de mejora, dada una solución inicial, se trata de aplicar transformaciones sobre esta de tal manera que mejore respecto de una determinada métrica.
          \item \emph{Redes Neuronales}: Apoyadas en la relación de cercanía entre nodos del grafo generado, en los últimos años se han propuesto métodos de resolución para problemas básicos mediante \emph{SOM}.
        \end{itemize}
    \end{itemize}
  \end{frame}

  \begin{frame}{Métodos de Resolución}
    \begin{itemize}
      \item Meta-Heurísticos:
        \begin{itemize}
          \item \emph{GRASP}: Estrategia combinada basada en generación de soluciones \emph{Greedy} para su posterior mejora mediante \emph{Local Search}.
          \item \emph{Simulated Annealing}: Para tratar de reducir la problemática de caer en mínimos locales, en las primeras etapas de ejecución del algoritmo se aceptan ciertas soluciones \say{contraproducentes} para después hacer más estricto dicho criterio conforme aumentan las iteraciones.
          \item \emph{Tabu Search}: Se permite alcanzar soluciones peores que la actual, siempre y cuando estas no se hayan visitado previamente (para ello se almacena la secuencia de últimas soluciones generadas).
          \item \emph{Variable Neighborhood Search}: En cada iteracción de la estrategia se utiliza una heurística diferente con la esperanza de que la combinación de ambas permita alcanzar resultados mejores.
          \item \emph{Large Neighborhood Search}: La generación de soluciones iniciales se basa en la recombinación y/o fusión de las soluciones obtenidas en anteriores iteraciones del algoritmo.
        \end{itemize}
    \end{itemize}
  \end{frame}

  \begin{frame}{Implementación y Resultados (I)}
    \begin{figure}[!hb]
      \centering
      \inputminted[frame=single]{python}{./code/minimal_solve.py}
    \end{figure}
  \end{frame}

  \begin{frame}{Implementación y Resultados (II)}
    \begin{table}[!ht]
      \centering
      \begin{tabu}{ | c | c | c | c |}
        \hline
        &\bfseries R & \bfseries A & \bfseries B
        \\\hline Finalización & 20.0\% & 38.09\% & 42.85\%
        \\\hline
      \end{tabu}
      \caption{Indicadores relativos al \emph{porcentaje de finalización} antes de las dos horas obtenido durante el experimento.}
      \label{table:ending_summary}
    \end{table}

    \begin{table}[!ht]
      \centering
      \begin{tabu}{ | c | c | c | c |}
        \hline
        &\bfseries R & \bfseries A & \bfseries B
        \\\hline Promedio & 50.18\% & 14.55\% & 10.34\%
        \\\hline Desviación Estándar & 29.08 & 10.68 & 10.02
        \\\hline
      \end{tabu}
      \caption{Indicadores relativos a la \emph{diferencia de costes} respecto del mejor valor conocido obtenido durante el experimento.}
      \label{table:costs_summary}
    \end{table}

    \begin{table}[!ht]
      \centering
      \begin{tabu}{ | c | c | c | c |}
        \hline
        &\bfseries R & \bfseries A & \bfseries B
        \\\hline Promedio & 94.85\% & 99.24\% & 98.82\%
        \\\hline Desviación Estándar & 6.43 & 1.62 & 1.79
        \\\hline
      \end{tabu}
      \caption{Indicadores relativos al \emph{nivel de servicio} obtenido durante el experimento.}
      \label{table:service_summary}
    \end{table}
  \end{frame}

  \begin{frame}{Conclusiones y Próximos Pasos}
    \begin{itemize}
      \item \emph{Conclusiones}:
        \begin{itemize}
          \item El estudio del problema \emph{Dial-a-Ride} representa un reto interesante, tanto por su complejidad desde el punto de vista teórico como del punto de vista de la implementación de métodos de resolución.

          \item La definición de una buena metodología de trabajo en proyectos científicos es de vital importancia para el correcto desarrollo de los mismos.

        \end{itemize}
      \item \emph{Próximos Pasos}:
        \begin{itemize}

          \item Profundización en el análisis y posible implementación de los métodos de resolución que representan el actual \emph{estado del arte} en la materia.

          \item Mejora de la implementación llevada a cabo, aumentando su escalabilidad de tal manera que sea utilizable en instancias de tamaño más realista.

        \end{itemize}
    \end{itemize}
  \end{frame}

  \begin{frame}
    \begin{center}
      \begin{huge}
        ¿Preguntas?
      \end{huge}
    \end{center}
	\end{frame}

\end{document}